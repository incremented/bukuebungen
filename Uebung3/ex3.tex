\section*{3. Übung}
\subsection*{Aufgabe 3.1}
Gegeben sei die Turingmaschine $M=(\{q_1, q_2, q_3\}, \{0,1\}, \{0,1,B\}, B, q_1, q_2, \delta)$ mit $\delta$ wie folgt:
\begin{center}
	\begin{tabular}{l|lll}
		$\delta$ & \hspace{0.6cm}0 & \hspace{0.6cm}1 & \hspace{0.6cm}B \\ \hline
		$q_1$ & ($q_3$, 1, N) & ($q_1$, 0, R) & ($q_2$, B, L) \\
		$q_3$ & ($q_1$, 0, L) & ($q_3$, 1, L) & ($q_1$, B, R)
	\end{tabular}\\
\end{center}
Berechnen Sie die Gödelnummer $\langle M \rangle$ von $M$ wie in der Vorlesung definiert.
\\

$\langle M \rangle = $ code(1) 11 code(2) 11 code(3) 11 code(4) 11 code(5) 11 code(6) 111
\\

Mit:
\\
$\delta$($q_{1}$, 0): code(1) = 0101000100100
\\
$\delta$($q_{1}$, 1): code(2) = 010010101000
\\
$\delta$($q_{1}$, B): code(3) = 01000100100010
\\
$\delta$($q_{3}$, 0): code(4) = 00010101010
\\
$\delta$($q_{3}$, 1): code(5) = 000100100010010
\\
$\delta$($q_{3}$, B): code(6) = 00010001010001000
\\


\subsection*{Aufgabe 3.2}
Sei $M = (Q, \Sigma, \Gamma, B, q_0, \bar{q}, \delta)$ eine 1-Band-TM, deren Speicherbedarf für eine Eingabe der Länge $n$ maximal $s(n)$ beträgt. Zeigen Sie: Wenn $M$ auf einer Eingabe $w$ der Länge $n$ hält, dann hält $M$ auf $w$ nach spätestenz $(\left| Q\right| -1) \cdot \left| \Gamma\right|^{s(n)} \cdot s(n)+1$ Schritten.\\\\

Der Term $(\left| Q\right| -1) \cdot \left| \Gamma\right|^{s(n)} \cdot s(n)$ beschreibt die Anzahl aller möglichen Konfigurationen
die eine TM erreichen kann für $|Q| - 1$ Zustände. Da wir 
annehmen, dass die TM auf der Eingabe hält, erreicht sie nach 
einer endlichen Anzahl von Schritten auch ihren Endzustand
welcher in dem obigen Term nicht berücksitigt wurde. Da jede
Konfiguration genau einem Schritt der TM entspricht heisst dies
aber auch, dass nach $(\left| Q\right| -1) \cdot \left| \Gamma\right|^{s(n)} \cdot s(n)$ vielen Schritten nur noch ein Schritt
maximal übrig ist um in den Endzustand zu gelangen (Unter der
Annahme, dass die TM hält). Deswegen beschreibt 
$(\left| Q\right| -1) \cdot \left| \Gamma\right|^{s(n)} \cdot s(n)+1$ die maximale Anzahl von Schritten nach denen die TM terminiert.
\\\\


In den folgenden Aufgaben ist es \textbf{nicht} notwendig, die Turingmaschinen explizit anzugeben. Eine Beschreibung ihrer Arbeitsweise und Laufzeit in den einzelnen Arbeitsschritten genügt.

\subsection*{Aufgabe 3.3}
Sei $L=\{w\#w \mid w \in \{0,1\}^*\}$ (über dem Alphabet $\Sigma = \{0,1,\#\})$.\\
a) Beschreiben Sie eine möglichst effiziente 1-Band-TM, die $L$ entscheidet. Analysieren Sie den Zeit- und Speicherbedarf der von ihnen entworfenen Maschine.\\

\begin{enumerate}
\item Die TM 'merkt' sich erste Zeichen und setzt dieses auf B. Ist das Zeichen ein B wird mit 0 terminiert, bei $\#$ wird zu 7. gesprungen.
\item Für 0 und 1 wird jeweils in einen eigenen Zustand gewechselt, die aber beide nach Rechts gehen, bis sie ein $\#$ lesen.
\item Die beiden Zustände wechseln jeweils in einen neuen Zustand, welcher nach rechts läuft, bis ein Zeichen ungleich $\#$ gelesen wird.
\item Dieses neue Zeichen wird mit dem gemerkten Zeichen verglichen. Sind sie nicht gleich wird mit 0 terminiert. Sind sie gleich, so wird das Zeichen auf $\#$ gesetzt.
\item Es wird nach links gelaufen bis das erste Zeichen (ungleich B) gelesen wird.
\item Ist das erste Zeichen nicht $\#$, wird zu 1. gesprungen.
\item Es wird nach rechts gegangen bis ein B gelesen wird. Wird dabei eine 0 oder 1 gelesen wird mit 0 terminiert. Wird ein B gelesen wird mit 1 terminiert.
\end{enumerate}
Falls die TM terminiert:\\
Für Länge($w$) $= n$ werden für jedes Zeichen von $w$ $n+1$ Schritte getan bis der Lesekopf auf dem Zeichen steht, mit welchem verglichen wird und anschließend muss der Lesekopf $n$ Schritte an den Anfang zurückgehen. Am Ende läuft die TM noch einmal $n+1$ Schritte über die $\#$ Zeichen und ein letztes B.\\
Insgesamt braucht die TM also $n*(2n+1)+n+1$ Schritte, die Laufzeit ist also $O(n^2)$.\\
Es werden alle Elemente von $w$ besucht, sowie das ursprüngliche $\#$ Zeichen, die $n$ Zeichen hinter $\#$ und gegebenenfalls noch ein B am Ende.\\
Zusammen also $2n+2$ Zellen.\\\\

\newpage

b) Beschreiben Sie eine möglichst effiziente 2-Band-TM, die $L$ entscheidet. Analysieren Sie den Zeit- und Speicherbedarf der von ihnen entworfenen Maschine.\\
\textbf{Hinweis:} Überlegen Sie sich zuerst, wie wie ein zweites Band die Erkennung eines Wortes in $L$ schneller machen kann.

\begin{enumerate}
	\item Das leere Wort terminiert mit 0. Beide Bände laufen nach Rechts bis auf Band 1 ein $\#$ gelesen wird. Dabei wird jedes Zeichen, welches auf Band 1 gelesen wird, auf Band 2 kopiert und anschließend gelöscht.
	\item Wird ein $\#$ gelesen löscht Band 1 dieses und geht einen Schritt nach Rechts.
	Band 2 geht nun an den Anfang zurück.
	\item Nun laufen beide Bänder nach Rechts und vergleichen die gelesenen Zeichen. Werden unterschiedliche Zeichen gelesen wird mit 0 terminiert.
	\item Lesen beide Bänder ein B wird mit 1 terminiert.
\end{enumerate}
Für Länge($w$) $=n$ läuft die TM zunächst mit beiden Bändern die $n$ Zeichen ab. Anschließend wird der Lesekopf auf Band 2 mit $n$ Schritten auf den Anfang zurückgesetzt. Am Ende werden erneut $n$ Schritte benötigt um die beiden Bänder zu vergleichen.\\
Das macht zusammen $3n$ Schritte, die Laufzeit beträgt also $O(n)$.\\
Band 1 wird einmal gesamt durchlaufen, was 2n+1 Zellen sind. Auf Band 2 werden nur die n Elemente von einem $w$ kopiert, welche auch vollständig durchlaufen werden.\\
Das macht einen Speicherverbrauch von 3n+1 Zellen.\\\\



\subsection*{Aufgabe 3.4}
Zeigen Sie, dass jede 1-Band-TM durch eine 1-Band-TM mit einseitig unendlichem Band, d.h., durch eine Turingmaschine, die die Position $p < 0$ nie benutzt, simuliert werden kann. Wie groß ist der Zeitverlust?\\\\

Wir benutzen eine zweite Spur um  die Positionen $p \leq 0$ zu simulieren.\\
An Position $p_0 = 0$ steht ein einzigartiges Zeichen auf beiden Spuren, um diese Position zu erkennen.\\
Zugriffe auf $p \leq 0$ werden zu $1+ \left| p \right|$.\\
$\delta$ wird in $\delta'$ geklont.\\
Solange der Lesekopf auf $p < 0$ steht wird Spur 2 ignoriert und einzig in Spur 1 gearbeitet.\\
Sobald das Sonderzeichen, welches Position 0 markiert, gelesen wird, wird zu $\delta'$ gewechselt.\\
$\delta'$ ignoriert Spur 1 und arbeitet in Spur 2. Zusätzlich werden die Schrittrichtungen gespiegelt (L wird zu R, R zu L).\\
Selbstverständlich wird hier beim Lesen vom Sonderzeichen wieder zu $\delta$ gewechselt.\\\\

Die Laufzeit ist hier nicht anders als bei einer beidseitig unendlichen TM (der Speicherverbrauch ist aber höher).