\section*{3. Übung}
\subsection*{Aufgabe 3.1}
Beschreiben Sie eine 1-Band-TM, die die Sprache $L = \{0^n1^n \mid n \in \mathbb{N}\}$ mit einem Zeitbedarf in $O(m \hspace{0,15cm} log \hspace{0,15cm} m)$ akzeptiert, wobei $m$ die Länge der Eingabe bezeichnet.\\
Es ist \textbf{nicht} notwendig die Turingmaschine explizit anzugeben. Eine Beschreibung ihrer Arbeitsweise und Laufzeit in den einzelnen Arbeitschritten genügt.
\subsection*{Aufgabe 3.2} Geben Sie das Programm einer Registermaschine zur Berechnung des Zweierlogarithmus $\lfloor $log$_2 \hspace{0,1cm} n \rfloor$ für eine Eingabe $n \in \mathbb{N}$ an. Erläutern Sie kurz seine Funktionsweise.\\

Wir speichern die Eingabe in $c(1)$, in $c(2)$ wird später die Ausgabe sein. $c(0)$ und $c(2)$ sind zunächst mit 0 initialisiert.
\begin{enumerate}
	\item LOAD 2
	\item CADD 1
	\item STORE 2
	\item LOAD 1
	\item CDIV 2
	\item STORE 1
	\item IF $c(0) > 0$ GOTO 1
	\item LOAD 2
	\item CSUB 1
	\item END
\end{enumerate}
Die Registermaschine berechnet wie viele mal man die Eingabe durch 2 teilen muss um auf 1 zu kommen.

\subsection*{Aufgabe 3.3}
Zeigen Sie, dass die Menge $\mathbb{N}^* = \bigcup_{n \in \mathbb{N}} \mathbb{N}^n$ der endlichen Wörter über den natürlichen Zahlen abzählbar ist.
\subsection*{Aufgabe 3.4}
Welche der folgenden Sprachen sind entscheidbar? Beweisen Sie die Korrektheit ihrer Antwort.\\
a) $H_{\leq 42} = \{\langle M \rangle w \mid M$ hält auf Eingabe $w$ und zwar nach höchstens 42 Schritten$\}$\\

Die Sprache $H_{\leq 42}$ ist entscheidbar. Wir kreiern eine 
TM $M_{H \leq 42}$ die wie folgt arbeitet. $M_{H \leq 42}$
 simuliert $\langle M \rangle$ mit der Eingabe $w$ und 
speichert zusätzlich einen Zähler welche die Anzahl der bereits
ausgeführten Schritte
zählt. Terminiert $M$ bevor der Zähler 42 erreicht, so akzeptiert
$M_{H \leq 42}$, andernfalls verwirft $M_{H \leq 42}$ nach 42
Schritten.\\

b) $H_{\geq 42} = \{\langle M \rangle w \mid M$ hält auf Eingabe $w$ und zwar nach mindestens 42 Schritten$\}$\\

Die Sprache $H_{\geq 42}$ ist nicht entscheidbar. Angenommen eine 
TM $M_{H\geq 42}$ würde existieren, welche $H_{\geq 42}$
entscheidet. Wir könnten dann eine TM $M_{H_{\epsilon}}$
konstruieren welche das Halteproblem $H_{\epsilon}$ entscheidet.
Hierzu würde $M_{H_{\epsilon}}$ aus der gegebenen TM $\langle M 
\rangle$ eine TM 
$M'$ berechnen, welche 42 Schritte nach rechts läuft und 
anschliessend die TM $M$ mit Eingabe $w$ simuliert. 
$M_{H_{\epsilon}}$ ruft also $M_{H\geq 42}$ mit $M'$ als 
Unterprogramm auf und übernimmt das Akzeptanzverhalten.


TODO: Korrektheit zeigen



