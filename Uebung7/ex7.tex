\section*{4. Übung}
\subsection*{Aufgabe 7.1}
Sei $k \in \mathbb{N}$ fest. Geben Sie für folgende RAM-Befehle jeweils ein äquivalentes RAM-Programm mit eingeschränktem Befehlssatz (siehe Folie 310) an, wobei Sie voraussetzen können, dass alle Register $j$ mit $j \ge k$ den Wert $c(j) = 0$ haben.\\
Verfügbare Befehle:\\
LOAD, STORE, CLOAD, CSUB, CADD, GOTO, IF $c(0) \neq 0$ THEN GOTO $j$, END
\begin{enumerate}[(a)]
	\item MULT 1, das heißt $c(0) := c(0) \cdot c(1)$\\
	Wir benutzen zwei Schleifen: Die innere addiert $c(1)$-mal 1 auf das Ergebnis, die äußere lässt dies $c(0)$ mal passieren.\\
	$k$ speichert das Ergebnis und ist anfangs 0
	\begin{enumerate}[1.]
		\item STORE $k+1$	 (Speichert das originale $c(0)$ um darüber zu iterieren)
		\item LOAD 1
		\item STORE $k+2$	 (Speichert das originale $c(1)$)
		\item STORE $k+3$	 (Speichert das originale $c(1)$ um darüber zu iterieren)
		\item LOAD $k+1$
		\item IF $c(0) \neq 0$ THEN GOTO 6	 (Wenn $c(1) = 0$ ist, sind wir mit der Multiplikation fertig)
		\item LOAD $k$
		\item END
		\item CSUB 1
		\item STORE $k+1$ 
		\item LOAD $k+2$	(Reset der kleinen Schleife)
		\item STORE $k+3$
		\item LOAD $k+3$
		\item IF $c(0) \neq 0$ THEN GOTO 16	(Wenn $c(k+3) = 0$ ist, sind wir mit der kleinen Schleife fertig)
		\item GOTO 5
		\item CSUB 1
		\item STORE $k+3$
		\item LOAD $k$
		\item CADD 1
		\item STORE $k$
		\item GOTO 13
	\end{enumerate}
	
	\item INDLOAD $i$
	INDLOAD: $c(0) := c(c(i))$
	\begin{enumerate}[ 1.]
		\item LOAD $i$
		\item IF $c(0) \neq 0$ GOTO 5
		\item LOAD 0
		\item END
		\item CSUB 1
		\item IF $c(0) \neq 0$ GOTO 9
		\item LOAD 1
		\item END
		\item CSUB
		\item IF $c(0) \neq 0$ GOTO 11
		\item LOAD 2
		\item END\\
		...
	\end{enumerate}
	\begin{enumerate}[j+1]
		\item CSUB
		\item IF $c(0) \neq 0$ GOTO j$+5$
		\item LOAD $k-1$
		\item END
		\item CSUB
		\item LOAD k
		\item END
	\end{enumerate}
\end{enumerate}

\subsection*{Aufgabe 7.2}
\begin{enumerate}[(a)]
	\item Ist das Problem, ob ein gegebenes LOOP-Programm zu einer Eingabe $x$ die Ausgabe $y$ berechnet, entscheidbar? Begründen Sie die Antwort.\\
	Per Definition terminiert ein LOOP-Programm immer. Darum kann das Programm mit der Eingabe $x$ einfach laufen lassen und das Ergebnis mit $y$ vergleichen. Das Problem ist somit entscheidbar.
	\item Ist das Problem, ob ein gegebenes WHILE-Programm zu einer Eingabe $x$ die Ausgabe $y$ berechnet eintscheidbar? Begründen Sie die Antwort.
\end{enumerate}
\subsection*{Aufgabe 7.3}
Für jedes LOOP-Programm $P$ sei $\langle P \rangle$ eine geeignete Kodierung von $P$ (ähnlich zu Gödelnummern für Turingmaschinen). Sei
$$A_{ \text{LOOP} }= \{\langle P \rangle \mid P \text{ gibt bei Eingabe 0 das Ergebnis 1 zurück}\}.$$
Welche der folgenden beziehungen zwischen $A_{\text{LOOP}}$ und dem Halteproblem $H$ für Turingmaschinen treffen zu?
\begin{enumerate}[(a)]
	\item $A_{\text{LOOP}} \leq H$
	\item $H \leq A_{\text{LOOP}}$
\end{enumerate}
Begründen Sie Ihre Antwort.
\subsection*{Aufgabe 7.4}
Zeigen Sie, dass $A(m+1,n) > A(m,n)$ für alle $m,n \in \mathbb{N}$ gilt. Sie dürfen dabei ohne Beweis verwenden, dass $A(m,n+1) > A(m,n)$.\\
\textbf{Bemerkung:} Als Zusatz können Sie bei Interesse versuchen, beide Ungleichungen zu beweisen.