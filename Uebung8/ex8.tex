\section*{8. Übung}
\subsection*{Aufgabe 8.1}

a)\\
Wir konstruieren eine NDTM $M$ welche wie folgt arbeitet:\\
\begin{enumerate}

\item Checke die Syntax der Eingabe, wenn fehlerhaft verwirf
\item Laufe von links nach rechts über $v$ und zähle mit einem Zähler auf einer separaten Spur
die Schritte. Speichere $v_i$ im aktuellen Zustannd, gehe nicht deterministisch in Zustand 4 oder laufe weiter
\item Laufe mit Hilfe des Zählers von der $\#$ bis $w_i$. Überprüfe folgendes:

    \begin{enumerate}

        \item $v_i \not= \#$, $w_i \not= B$ und $v_i \not=w_i$ $\Rightarrow$ akzeptiere
        \item $v_i = \#$, $w_i \not= B$ $\Rightarrow$ akzeptiere
        \item $v_i \not=$, $w_i=B$ $\Rightarrow$ akzeptiere
    \end{enumerate}

\end{enumerate}

Bezüglich der Laufzeit:\\
Damit die Laufzeit nicht in $O(n^2)$ ist, müssen wir den Zähler auf der 2. Spur bei jedem Schritt mitziehen.
Dies kostet uns pro Schritt $O(log(n))$\\
Daher ergibt sich die Gesamtlaufzeit von $O(n \cdot log(n))$\\

b)\\
Nein, da wir für Zugehörigkeit in $L'$ jede Position überpfüfen müssen.

\newpage
\subsection*{Aufgabe 8.2}

a)\\
Graph 1:\\
Ist 2-Färbbar, da eine Abbildung c wie folgt existiert:\\

 \[
     c: V \rightarrow \{1,2\}, c(v)=\left\{\begin{array}{ll} 1, & $für$ \:v1, v2, v5, v7 \\
         2, & $für$ \: v3, v4, v6\end{array}\right. .
  \]

Graph2:\\
Graph 2 ist nicht 2-Färbbar, da er einen Zykel ungerader Länge enthält und somit immer 2
gleich gefärbte Knoten nebeneinander liegen.\\

b)\\
Idee:\\
Suche nach Zyklen ungerader Länge im Graph. Existiert solch ein Zyklus, ist der Graph
nicht 2-Färbbar. Da ein Graph genau dann bipartit ist, wenn er keinen solchen Zykel
enthält, reicht es einen Algorithmus anzugeben, der die Bipartitheit eines
Graphen entscheidet. Wir nutzen hierfür eine leichte Abwandlung der Tiefensuche wie folgt:\\

Algorithmus:  2 leere Mengen $V_{1} = V_{2} = \varnothing$\\
Führe Tiefensuche auf Knoten $v$ aus.
\begin{enumerate}
\item Ordne $v$ der Menge $V_1$ zu
\item Ordne jedem Nachbarknoten alle Knoten aus $V_1$ der Menge $V_2$ zu und umgekehrt, sofern
diese noch nicht besucht wurden.
\item Ist ein Knoten $v$ nun einer Menge zugeordnet und es wir festgestellt, dass ein 
Nachbarknoten von $v$ bereits besucht ist und ebenfalls in der gleichen Menge wie $v$ ist,
Verwerfe!
\item Wurde kein solch obiger Konflikt gefunden, ist der Graph bipartit, also auch 2-Färbbar.
\end{enumerate}

Korrektheit:\\

Fall1 G ist bipartit:\\
Ordne ersten Knoten $v$ der Menge $V_1$ zu und verfahre nach dem Schema der Tiefensuche
für alle Knoten, so dass die Nachbarknoten immer der jeweils anderen Menge ($V_1$ oder $V_2$)
des Vaterknoten zugeordnet werden, sofern sie nicht besucht wurden.\\
$\Rightarrow$ Da G bipartit ist, bekommt jeder Knoten eine vom Vaterknoten verschiedene Menge zugewiesen.\\
$\Rightarrow$ Der Algorithmus gibt aus G ist bipartit\\
$\Rightarrow$ G ist 2-Färbbar\\

Fall2 G ist nicht bipartit:\\
$\Rightarrow$ Ordne den ersten Knoten $V_1$ zu\\
$\Rightarrow$ Verfahre nach dem Schema der Tiefensuche\\
$\Rightarrow$ Nach $n-1$ korrekten Zuweisungen, wird die in Schritt 3) aufgestellte\\
Bedingung für den $n-ten$ Knoten, nach dess Besuch verletzt.\\
$\Rightarrow$ Es existier eine Kante zwischen 2 Knoten in der gleichen Menge\\
$\Rightarrow$ Der ist nicht bipartit\\
$\Rightarrow$ G ist nicht 2-Färbbar\\

Zu zeigen: 2-Colorability liegt in P\\
Der Algorithmus beruht auf der normalen Tiefensuche. Die zusätzlichen Schritte benötigen alle
eine konstante Laufzeit. Die Gesamtlaufzeit erschliesst sich also aus der Menge aller Knoten
und Kanten des Graphen und liegt bei $O(|V| + |E|)$

\subsection*{Aufgabe 8.3}

Zu zeigen: 3-Colorability liegt in NP\\

Beweis:\\
Wir beschreiben eine NDTM $M$ mit $L(M)=$ 3-Colorability. $M$ arbeitet wie folgt:\\
\begin{enumerate}
\item Falls die Eingabe kein Graph/Adjazenzmatrix ist verwerfe
\item Sei $G=(V,E)$. Sei $n$ die Anzahl der Knoten, so dass $V = \{1,...,n\}$. 
$M$ schreibt hinter die Eingabe den String $\#^{n}$. Der Kopf bewegt sich unter das erste $\#$
\item (nicht deterministisch) $M$ läuft von links nach rechts über den String $\#^{n}$ und 
ersetzt ihn nicht-deterministisch durch einen String aus $\{1, 2, 3\}^{n}$
(Jeder Knoten bekommt eine Farbe zugewiesen)
\item Prüfe für alle Kanten $e \in E$, ob die 3-Colorability Eigenschaft gilt und 
akzeptiere entsprechend.
\end{enumerate}

Wir zeigen $L(M)=$ 3-Colorability:\\
Sei die Eingabe ein gültiger Graph G und sei $G \in $ 3-Colorability.
In diesem Fall gibt es mindestens eine Farbkombination welche in Schritt 4 akzeptiert wird.\\
Falls die Eingabe also in 3-Colorability enthalten ist, gibt es einen akzeptierenden
Rechenweg polynomiell beschränkter Länge.\\
Falls die Eingabe nicht in 3-Colorability enthalten ist, gibt es auch keinen
akzeptierenden Rechenweg.\\

Fazit:\\
$M$ erkennt die Sprache 3-Colorability und hat eine polynomiell beschränkte Worst-Case-Laufzeit\\
$\Rightarrow$ 3-Colorability $\in$ NP


\subsection*{Aufgabe 8.4}





