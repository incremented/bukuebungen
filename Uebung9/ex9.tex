\section*{9. Übung}
\subsection*{Aufgabe 9.1}
Zeigen Sie, jeweils mit Hilfe eines Polynomialzeitverifizierers, dass die folgenden Entscheidungsprobleme in NP sind. Beschreiben Sie dazu im Detail die Kodierung und die Länge das Zertifikats sowie die Arbeitsweise und die Laufzeit des Verifizierers (eine Beschreibung ähnlich zu Folie 418 ist \textbf{nicht} hinreichend detailliert)
\begin{enumerate}[(a)]
	\item $\textsc{MaxSpanTree} = \{ (G,c) \mid c \in \mathbb{N} $ und $G $ ist ein Graph mit gewichteten Kanten und hat einen Spannbaum mit Kosten $ \geq c \}$
	
	$\textsc{MaxSpanTree}$ ist in NP:\\
	Es kodiert das Zertifikat $z \in \{ 0,1 \}^{|E|}$ welche Kanten $z_i \in E=\{e_1, \dots, e_n \}$ in unserem Spannbaum enthalten sind. Dabei gilt $z_i = 1$, wenn $z_i$ im Spannbaum enthalten ist.\\
	Es gilt auch $|z| = |E|$, die Eingabelänge ist somit polynomiell.\\
	Der Verifizierer prüft nun ob der Graph zusammenhängend und $|V|-1$ Stellen gleich 1 sind, also ob der durch $z$ kodierte Baum $T$ ein Spannbaum ist. Sollten diese Bedingungen nicht erfüllt sein wird die Eingabe verworfen.\\
	Anschließend prüfen wir, ob $T$ $V$ überdeckt.\\
	Ist auch das erfüllt, berechnen wir die Summe der Kantengewichte von $T$ $w(T)$, und prüfen $w(T) \geq c$. Ist dies erfüllt akzeptieren wir die Eingabe, ansonsten wird sie verworfen.\\
	
	Laufzeit:\\
	Prüfen der Einträge von $z$: $\mathcal{O}(|E|\text{log}(|E|))$\\
	Zusammenhang von $z$: Nach VL $\mathcal{O}(|V|^2)$\\
	Überdeckungsprüfung: Wir prüfen für alle Knoten alle sonstigen $|V|-1$ Kanten, erhalten also $\mathcal{O}(|V|^2)$\\
	Addieren der Kantengewichte und Vergleich mit $c$: Wir addieren $|V|-1$ Werte und erhalten $\mathcal{O}(|V||E|)$ als Laufzeit der Summierung und $\mathcal{O}(|V|c)$ als Komplexität des Vergleiches.\\
	Da alle Schritte in polynomieller Zeit (nach Eingabelänge) ausgeführt werden folgt die Behauptung.\\
	 
	\item $\textsc{Composite} = \{ w \in \{0,1\}^* \mid w $ ist die binäre Kodierung einer Zahl $k \in \mathbb{N} $ und $k$ ist keine Primzahl$ \}$
	
	$\textsc{Composite}$ ist in NP:\\
	Wir wählen als Zertifikat eine binäre Kodierung von $a \geq 2$ und $b \geq 2$ mit $k=ab$, zB also indem wir eine 0 vor jede Stelle der Binärdarstellung von $a$ schreiben, und eine 1 vor jede Stelle von $b$. Die Eingabelänge ist also polynomiell in der Eingabelänge.\\
	Ist $w=k$ keine (binär codierte) Primzahl, so können wir sie entsprechend in $a$ und $b$ zerlegen und so ein wo oben beschriebenes Zertifikat erstellen.\\
	Der Verifizierer prüft $a, b \neq 0$, und vergleicht danach $k=ab$.\\
	
	Laufzeit:\\
	Mit Ü7.1 können wir die Multiplikation in polynomieller Zeit ausführen.\\
	Zu prüfen ob $k=ab$ braucht nur $ \mathcal{O} (n \cdot \text{log }(k))$ (mit $n$ als Eingabelänge).\\
	Da alle Schritte in polynomieller Zeit (nach Eingabelänge) ausgeführt werden gilt die Behauptung.
	
	
	\item $\textsc{Graphisomorphie} = \{ G_1 \# G_2 \mid G_1, G_2 \text{ sind Graphen, }G_1 \text{ ist isomorph }G_2 \}$\\
	Zwei Graphen $G_1 = (V_1, E_1)$ und $G_2 = (V_2, E_2)$ sind isomorph, falls es eine bijektive Abbildung $f: V_1 \rightarrow V_2$ gibt, so dass $(v_i, v_j) \in E_1 \Leftrightarrow (f(v_i)), f(v_j)) \in E_2$ für alle $v_i, v_j \in V$ gilt.
	
	$\textsc{Graphisomorphie}$ ist in NP:\\
	Wir schaffen ein Zertifikat $z$, welches die Abbildung $f$ von $V_1 = \{ v_{1,1}, \dots, v_{1,n} \}$ nach $V_2 = \{ v_{2,1}, \dots, v_{2,n} \}$ als Funktion $g$ kodiert, mit $g(a)=b \Leftrightarrow f(v_{1,a})=v_{2,b}$.\\
	Die Kodierung sieht dabei wie folgt aus:\\
	Für einen Graphen $G=(V,E)$ schreiben wir $$\text{bin($n$)\#code($e_1$)\#...\#code($e_m$)}$$ (wobei \# für ein einzigartig genutztes Symbol steht).\\
	Hierbei gilt $$\text{code($e_i$) = code(($v_s, v_t$)) = (bin($s$), bin($t$)) und $s \leq t$}.$$
	Die Abbildung $f$ wird nun als $$\text{bin(1), bin(g(1))\#bin(2), bin(g(2))\#\dots\#bin(n), bin(g(n))}$$
	kodiert.\\
	Wie in (b) können wir die einzelnen Paare durch 0 und 1 als Präfix leicht voneinander trennen. \\
	Wir erhalten (maximal) $|V|$ Zuordnungen mit je $2\text{log}(|V|)$ Stellen, demnach ist das Zertifikat polynomiell in der Eingabelänge.\\
	Unser Verifizierer prüft nun $|V_1|=|V_2|$ indem er den ersten Eintrag von code($G_1$) und code($G_2$) vergleicht ($\mathcal{O}|(\text{gesamte Eingabe}|^2)$).\\
	Außerderm werden alle Knoten $v_{1,k}$ durch $f(v_{1,k}), k \in [1,n]$ in der Kodierung $G_1$ ersetzt, indem $k$ durch $g(k)$ ersetzt wird. Die Suche nach der entsprechenden Kodierung kann dabei in $\mathcal{O}(|z|\text{log } k)$ geschehen.\\
	Da alle Kanten welche inzident sind ersetzt werden müssen erhalten wir eine Laufzeit von $\mathcal{O}(|E|\cdot|z|\cdot\text{log }|V|), \text{log }k \leq \text{log }|V|$.\\
	Wir prüfen, ob im modifizierten Graphen der Index des ersten Knotens $\leq$ Index des zweiten Knotens ist, und tauschen diese ansonsten (in $\mathcal{O}(|E|\cdot \text{log } |V|))$.\\
	Schlussendlich wird überprüft, ob alle Kanten des modifizierten Graphen auch in $E_2$ liegen (und umgekehrt), indem wir für jede Kante die modifizierte Eingabe durchlaufen (in $\mathcal{O}(|E|^2\cdot| \text{Eingabe}|\cdot \text{log }|V|)$)\\
	Da alle Schritte in polynomieller Zeit (nach Eingabelänge) erfolgen gilt die Behauptung.
	
\newpage
	
\end{enumerate}
\subsection*{Aufgabe 9.2}
Seien BPP und TSP die in der Vorlesung vorgestellten Optimierungsprobleme. Zeigen Sie die folgenden Aussagen:
\begin{enumerate}[(a)]
	\item Falls BPP-E in P ist, so kann auch BPP in polynomieller Zeit gelöst werden.
	\item Falls TSP-E in P ist, so kann auch TSP in polynomieller Zeit gelöst werden.
\end{enumerate}

a)\\
Angenommen die Entscheidungsvariante des BPP $EV \in P$, dann existiert ein Algorithmus $A_{ENT}$, der
EV in P entscheidet. Wir beschreiben nun einen Algorithmus welcher mithilfe von $A_{ENT}$ das BPP löst.
\begin{enumerate}

	\item Finde durch Binärsuche die optimale Anzahl an Behältern $b_{Opt}$. Dies ist möglich, da $A_{ENT}$
		existiert und man sich sukzessiv an den optimalen Wert annähern kann. Laufzeit: $O(log(n))$
	\item Finde die Aufteilung auf die Behälter
	\begin{enumerate}
		\item Iteriere über alle Paare $(x,y)$ von Objekten
		\item "klebe" $x$ und $y$ zusammen und prüfe mit $A_{ENT}$ ob immer noch eine Verteilung
			auf $b_{Opt}$ exisitiert
		\item Falls ja: lasse x und y zusammen und rufe diesen Algorithmus hier rekursiv auf mit den
			zusammengeklebten Objekten, solange bis die Anzahl aller Objekte $= b_{Opt}$\\
			Falls nein: wähle nächstes Paar
	\end{enumerate}
\end{enumerate}

Laufzeit:\\
Pro Schritt des Algorithmus wird die Anzahl der Objekte um eins reduziert. Es entstehen $n - b_{Opt}$
Rekursionen. Pro Schritt haben wir höchstens $n^2$ viele Aufrufe $\Rightarrow$ Der Algorithmus ist in P \\

b)\\
Angenommen die Entscheidungsvariante $EV$ des TSP ist in P, dann existiert ein Algorithmus $A_{ENT}$,
der $EV$ in polynomialzeit entscheidet. Wir beschrieben nun einen Algorithmus welcher mithilfe von $A_{ENT}$
das TSP löst.
\begin{enumerate}
	\item Eingabe ist ein Graph $G(V,E)$
	\item Bestimme zunächst wieder $k_{Opt}$
	\item Der Algorithmus läuft über die Eingabe und löscht eine Kante $e \in E$ und prüft mittels
		$A_{ENT}$, ob es eine optimale Tour der Länge $k_{Opt}$ gibt.
		\begin{enumerate}
			\item Falls Ja: Wähle eine neue Kante $e' \in E$ mit $e' \not= e$ und beginne von vorne
			\item Falls Nein: Die Kante is notwendig für jede optimale Tour. Mache das Löschen dieser
				Kante rückgängig und wähle eine andere.
		\end{enumerate}
\end{enumerate}

Laufzeit:\\
Es wird genau einmal über jede Kante iteriert. Da die Entscheidungsvariante polynomiell lösbar ist bleiben 
wir auch mit diesem Algorithmus in einer polynomiellen Laufzeit.

\subsection*{Aufgabe 9.3}
Betrachten Sie folgendes Entscheidungsproblem:
\begin{framed}
	\textsc{DoubleSat}\\
	\textbf{Eingabe:} Eine aussagenlogische Formel $\varphi$ in KNF.\\
	\textbf{Ausgabe:} Ja gdw. es mindestens \textbf{zwei} erfüllende Belegungen für $\varphi$ gibt.
\end{framed}
Zeigen Sie, dass \textsc{DoubleSat} NP-vollständig ist.\\

Wir reduzieren \textsc{DoubleSat} auf \textsc{Sat}:\\
Wir fügen für eine Eingabe $X(x_1, \dots, x_n)$ eine Variable $y$ ein, wobei für die Ausgabe gilt $$X'(x_1, \cdots, x_n, y) = X(x_1, \cdots, x_n) \land (y \lor \bar{y}).$$
Wenn $X$ \textsc{Sat} $\notin$ \textsc{Sat}, $X$ also nicht erfüllbar ist, kann offensichtlicherweise auch $X'$ nicht erfüllbar sein und somit gilt auch $X' \notin \textsc{DoubleSat}$.\\
Wenn aber $X \in \textsc{Sat}$, dann kann man $X'$ lösen, indem man $y=0$ oder $y=1$ setzt. Man hat also zwei erfüllende Belegungen, und somit ist $X' \in \textsc{DoubleSat}$.\\
$\Rightarrow \textsc{Sat} \leq_p \textsc{DoubleSat}$\\
$\Rightarrow$ \textsc{DoubleSat} ist NP-vollständig.