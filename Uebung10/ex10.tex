%!TEX root = main.tex
\section*{10. Übung}
\subsection*{Aufgabe 10.1}
a) Entscheidungsvariante:\\
Einagbe: $m$ Maschinen, $n$ Job mit Laufzeiten $p_1,...,p_n$, und $k \in \mathbb{N}$\\
Ausgabe: Existiert ein Schedule mit einem Makespan $\le k$?\\

b)\\
Zu zeigen SUBSETSUM $\le_p$ $MAKESPANSCHEDULING_{ENT}$\\
Wir beschreiben eine Funktion f wie folgt:

\begin{itemize}
	\item Falls x nicht der Form $a_1,...,a_n \in \mathbb{N}$, $b \in \mathbb{N}$ ist, so ist $f(x) = 0$.
	\item Kreiere als Eingabe für $MAKESPANSCHEDULING_{ENT}$ n Jobs mit Laufzeit $p_i = a_i$.
	\item Um die Jobs einfach aufteilen zu können setzen wir die Anzahl der Maschinen für $MAKESPANSCHEDULING_{ENT}$ auf 2.
	\item Um zu garantieren, dass die Gesamtlaufzeit aller Jobs durch 2 teilbar ist kreieren wir nun 2 weitere Jobs $p_{n+1}$ und 
		$p_{n+2}$ wobei $p_{n+1}$ die Laufzeit $2S - b$ hat und $p_{n+2}$ die Laufzeit $S + b$ wobei $S = \sum_{i=1}^{n} a_i$.
	\item Als letztes setzen wir noch den geforderten MakeSpan auf $2*S$, da wir garantieren können, dass es immer eine Aufteilung
		auf die 2 Maschinen gibt, so dass der MakeSpan erfüllt wird. (Hierzu mehr in der Korrektheit)
\end{itemize}
Die Funktion ist polynomiell berechenbar, da lediglich die Summe $S$ berechnet werden muss sowie die 2 zusätzlichen Jobs.\\

Korrektheit: Sei $x \in \Sigma^{*}$

\begin{itemize}
	\item Wenn $x$ ist nicht der Form $a_1,...,a_n \in \mathbb{N}$, $b \in \mathbb{N}$ ist $\Rightarrow$ $f(x) = 0$ $\Rightarrow$ 
		$f(x) \not\in $ $MAKESPANSCHEDULING_{ENT}$
	\item Fall 1: Sei $x = a_1,...,a_n,b \in$ SUBSETSUM $\Rightarrow$ kreiere die Jobs $p_i$ für $i \in \{1,...,n+2\}$
	\item Wir machen uns nun die Eigenschaft zu nutze, dass die Summe einer Teilmenge $A'$ von $\{a_1,...,a_n\}$ gleich $b$ ist.
		Wir teilen nun jeden Job $p_i$ für $i \in \{1,...,n\}$ der Maschine $1$ zu, wenn $p_i \in A'$. Ansonsten wird $p_i$ Maschine $2$ zugeordnet.
	\item Es bleiben noch die Jobs $p_{n+1}$ und $p_{n+2}$. Wir ordnen $p_{n+1}$ Maschine $1$ zu. Maschine $1$ hat also die Laufzeit
		$2*S - b + \sum_{i \in A'} p_i$ was genau $2*S$ ist. Es bleibt noch $p_{n+2}$ welchen wir Maschine $2$ zuordnen. 
		Die Laufzeit hier berechnet sich wie folt: $S + b + \sum_{i \in A \setminus A'}p_i$ was ebenfalls $2*S$ ist.
	\item $f(x) \in $ $MAKESPANSCHEDULING_{ENT}$
	\item Fall 2: Sei $x = a_1,...,a_n,b \not\in$ SUBSETSUM $\Rightarrow$ kriere die $n+2$ jobs.
	\item Es existiert keine Menge $A'$ deren Summe gleich $b$ ist. $\Rightarrow$ Auf Grund unserer Konstruktion der Laufzeiten von
		$p_{n+1}$ und $p_{n+2}$ existiert kein Schedule der genau den geforderten MakeSpan hat.
	\item $\Rightarrow$ $f(x) \not\in$ $MAKESPANSCHEDULING_{ENT}$
\end{itemize}

\newpage
\subsection*{Aufgabe 10.2}
Zu zeigen 3-Partition is NP-vollständig.\\

Wir zeigen PARTITION $\le_p$ 3-PARTITION\\

Wir beschreiben eine Funktion f wie folgt:

\begin{enumerate}
	\item $f(x) = x$ gdw. x nicht die Form $\{a_1, ..., a_n\} \in \mathbb{N}$ hat.
	\item Ist $x = \{a_1, ..., a_n\} \in \mathbb{N}$, dann: $f(x) = x' = x \cup \{\lceil\frac{A}{2}\rceil\}$ wobei $A = \sum_{i=1}^{n}{a_i}$
\end{enumerate}

Die Funktion is offensichtlich in polynomieller Zeit berechenbar, da wir lediglich einmal über die Eingabe iterieren müssen um 
die Summe zu bilden.\\

Korrektheit: Sei $x \in \Sigma^{*}$
\begin{itemize}
	\item $x$ ist nicht der Form $\{a_1,...,a_n\} \in \mathbb{N}$, dann ist auch $f(x)$ nicht dieser Form $\Rightarrow$ $f(x) \not\in $
		3-PARTITION.
	\item Fall 1: $x = \{a_1,...,a_n\} \in \mathbb{N}$, und $x \in $ PARTITION $\Rightarrow$\\
		 $\exists k \subseteq \{1,...,n\}$ mit $\sum_{i \in k} a_i = \sum_{i \in \{1,...,n\} \setminus k} a_i$ 
		 und die Menge $f(x) = x'$ ist deswegen 3-Partitionierbar, da wir
		$\{\frac{A}{2}\}$ zu $x$ hinzufügen. $\Rightarrow$ $f(x) \in$ 3-PARTITION
	\item Fall 2: Sei $x \not\in$ PARTITION $\Rightarrow$ Es existiert keine Menge $k \subseteq \{1,...,n\}$
		mit $\sum_{i \in k} a_i = \sum_{i \in \{1,...,n\} \setminus k} a_i$ $\Rightarrow$ Die Menge $f(x) = x'$ enthält bereits 
		$\{\frac{A}{2}\}$. Da $\{\frac{A}{2}\}$ als einziges Objekt einer Partition zugeordnet werden muss, (Auf Grund seiner Grösse),
		muss die Menge $x' \setminus \{\frac{A}{2}\} \in $ PARTITION sein. Dies ist ein Widerspruch zur Annahme oben.
\end{itemize}
Damit ist die Korrektheit für jede Eingabe gezeigt.\\
Es bleibt zu zeigen, dass 3-PARTITION $\in NP$.\\

Zertifikat:\\
Ein Zertifikat $y$ wäre ein Wort $y \in \{0, 1\}^k$ der Form $(t_{11},...,t_{1n}), (t_{21},...,t_{2n}), (t_{31},...,t_{3n})$ wobei
$t_{1i} = 1 \Leftrightarrow a_i \in I$, $t_{2i} = 1 \Leftrightarrow a_i \in J$, und $t_{3i} = 1 \Leftrightarrow a_i \in K$. (Codierung einer 3-Partition für die Eingabe $x = \{x = (a_1,...,a_n)\}$)\\

Verifizierer:
\begin{enumerate}
	\item Falls $y$ nicht in der obigen Form vorliegt verwirf.
	\item Überprüfe ob $\sum_{i = 0}^{n} t_{1i} = \sum_{i=0}^{n} t_{2i} = \sum_{i=0}^{n} t_{3i}$. Falls nein, verwirf. Ansonsten 
	akzeptiere.
\end{enumerate}

$\Rightarrow$ Die Gesamtlaufzeit ist polynomiell\\
$\Rightarrow$ 3-PARTITION $\in$ NP\\
$\Rightarrow$ 3-PARTITION ist NP-hard.\\

$\Rightarrow$ 3-PARTITION ist NP-Vollständig


\subsection*{Aufgabe 10.3}
