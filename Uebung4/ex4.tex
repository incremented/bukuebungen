\section*{4. Übung}
\subsection*{Aufgabe 4.1}
Sei $$L_\text{self} = \{\langle M \rangle \mid M \text{ verwirft } \langle M \rangle \}.$$
Zeigen Sie durch Diagonalisierung, dass $L_\text{self}$ nicht entscheidbar ist.\

Beweis durch Widerspruch. Wir nehmen an, dass $L_\text{self}$
entscheidbar ist. Dann gibt es eine TM $M_{L_{\text{self}}}$ die
$L_\text{self}$ entscheidet. Mit Hilfe von $M_{L_{\text{self}}}$
konstruieren wir nun eine TM $M_{\bar{D}}$ welche die
Diagonalsprache entscheidet was zu einem Widerspruch führt. 
$M_{\bar{D}}$ arbeitet wie folgt:

\begin{enumerate}
	\item Berechne zur Eingabe $w$ das $i$, so dass gilt $w = w_i$
	\item Berechne nun die Gödelnummer der i-ten TM 
	$\langle M_i \rangle$
	\item Starte $M_{L_{\text{self}}}$ mit Eingabe 
	$\langle M_i \rangle$ und simuliere.
	\item Übernehme das akzeptansverhalten von $M_{L_{\text{self}}}$
\end{enumerate}

Korrektheit:\\

Sei $w \in \bar{D} \implies$ $M_{L_{\text{self}}}$ akzeptiert
$\langle M_i \rangle$ und hält\\
$\implies M_{\bar{D}}$ akzeptiert $w$\\

Sei $w \not\in \bar{D} \implies M_{L_{\text{self}}}$ verwirft
$\langle M_i \rangle$ und hält\\
$\implies M_{\bar{D}}$ akzeptiert $w$ nicht\.\

\subsection*{Aufgabe 4.2}
Zeigen oder widerlegen Sie, dass folgende Probleme entscheidbar sind. Sie können gegebenenfalls den Satz von Rice verwenden.
\begin{enumerate}[(a)]
	\item $\textbf{H}_\text{never}$: Gegeben eine TM $M$, stoppt $M$ auf keiner Eingabe?\\\\
	Nicht entscheidbar.\\
	Sei $\mathcal{S} = \{f_M \mid f_M(x) = \perp \forall x \in \Sigma^{*} \}$, $\emptyset \subsetneq \mathcal{S} \subsetneq \mathcal{R} $.\\
	Dann ist $L(\mathcal{S}) = \{ \langle M \rangle \mid M \text{ berechnet eine Funktion aus } \mathcal{S} \}$\\
	= $\{ \langle M \rangle \mid M \text{ hält nicht auf } x \hspace{0.1cm}\forall x \in \Sigma^{*} \}$\\
	$=H_{\text{never}}$\\
	$\overset{\text{S.v. Rice}}{\implies} H_{\text{never}}$ ist nicht entscheidbar.\\
	
	\item $\textbf{S}_{15}$: Gegeben eine TM $M$, besucht $M$ bei Eingabe 101 den Zustand $q_{15}$?
	
	\item $\textbf{I}_\mathbb{P}$: Gegeben eine TM $M$, gilt $L(M)= \mathbb{P}$? Hierbei sei $\mathbb{P} = \{\text{bin}(p) \mid p \text{ ist Primzahl} \}$ die Menge der Binärdarstellungen der Primzahlen.\\\\
	Wir erschaffen eine Turingmaschine, welche die Eingabe auf restlose Teilbarkeit durch 2 bis $n-1$ prüft.
	Dazu wählen wir eine 3-spurige TM.\\
	Auf die erste Spur wird die Eingabe geschrieben.\\
	Die zweite Spur hält den aktuellen (in binär codierten) Teiler.\\
	Auf der dritten Spur findet die Berechnung statt.\\
	Anfangs schreiben wir auf die zweite Spur eine 1 und prüfen, ob die Eingabe $ \ge 2$ ist. Bei 0 oder 1 wird rejected.\\
	(*) Der Algorithmus geht nun wie folgt vor:\\
	Spur 1 wird auf Spur 3 kopiert und Spur 2 um 1 vergrößert. Sind Spuren 1 und 2 nun gleich wird die Zahl akzeptiert.\\
	Auf Spur 3 wird nun ein beliebiger Divisions-Algorithmus ausgeführt. Erhält die Maschine dabei einen Rest von 0 wird die Zahl rejected. Ansonsten wird Spur 3 gesäubert, der Lesekopf auf die erste Position zurückgesetzt und der Algorithmus geht von (*) an wieder los.\\
	$\textbf{I}_\mathbb{P}$ ist somit entscheidbar.\\
	
	\item $\textbf{L}_\text{comp}$: Gegeben zwei TMen $M_1$ und $M_2$, gilt $L(M_1) = \overline{L(M_2)}$?\\
	Wir versuchen mit der Unterprogrammtechnik die Unentscheidbarkeit zu beweisen.
	
	$$\text{L}_\text{comp} := \{ \langle M_1 \rangle \langle M_2 \rangle \mid L(M_1) = \overline{L(M_2)} \}$$
	
	$$f : x \in H \Leftrightarrow f(x) \in \text{L}_\text{comp}$$
	$$\rightarrow x = \langle M \rangle w : f(x) = \langle M^{*} \rangle \langle M' \rangle$$
	$$\rightarrow x \neq \langle M \rangle w : f (x) = x$$
$M^*$ ist Gödelnummer einer TM welche die Eingabe löscht, $w$ auf das Band schreibt, $M$ auf $w$ simuliert und w akzeptiert wenn $M$ auf $w$ hält.\\
$M'$ ist Gödelnummer einer TM welche alle Eingaben rejected.\\
Da sowohl $M^*$ und $M'$ berechnet werden können ist auch $f$ berechenbar.\\
Korrektheit:\\
$x \neq \langle M \rangle w : x \notin H , f(x) \notin \text{L}_\text{comp}$\\
$x= \langle M \rangle w $\\
Für $ x \in H$:\\
$\Rightarrow M$ hält auf $w$\\
$\Rightarrow M^*$ akzeptiert alle Eingaben.\\
$\Rightarrow M^*$ akzeptiert das Komplement der Sprache die $M'$ akzeptiert.\\
$\Rightarrow \langle M^* \rangle \langle M' \rangle \in \text{L}_\text{comp}$\\
$\Rightarrow f(x) \in \text{L}_\text{comp}$\\
Für $x \notin H$:\\
$\Rightarrow M$ hält nicht auf $w$\\
$\Rightarrow M^*$ akzeptiert keine Eingabe\\
$\Rightarrow M^*$ akzeptiert das Komplement der Sprache die $M'$ akzeptiert nicht.\\
$\Rightarrow \langle M^* \rangle \langle M' \rangle \notin \text{L}_\text{comp}$\\
$\Rightarrow f(x) \notin \text{L}_\text{comp}$\\
	
$x \in H \Leftrightarrow f(x) \in \text{L}_\text{comp}$\\
$\Rightarrow \text{L}_\text{comp}$ ist unentscheidbar.
	
	
\end{enumerate}

\subsection*{Aufgabe 4.3}
Für eine Turingmaschine $M$ über dem Eingabealphabet $\Sigma = \{0,1\}$ und einem Wort $w \in \Sigma^*$ sei $M_w^*$ eine Turingmaschine, die bei Eingabe $\epsilon$ zunächst das Wort $w$ auf das Band schreibt und dann $M$ auf $w$ simuliert. Bei anderen Eingaben darf sich $M_w^*$ beliebig verhalten.
\begin{enumerate}[(a)]
	\item Geben Sie eine \textbf{formale} Definition für $M_w^*$ an.
	\item Beschreiben Sie grob die Funktionsweise einer Turingmaschine $N$, die bei Eingabe $\langle M \rangle w$ die Gödelnummer von $M_w^*$ berechnet. Sollte die Eingabe nicht das vorgegebene Format haben darf sich die Turingmaschine $N$ beliebig verhalten.\\
	\textbf{Hinweis:} Sie können für $N$  eine Mehrband-TM verwenden.
\end{enumerate}
\textbf{Bemerkung:} Diese Aufgabe ist Teil des Beweises für die Unentscheidbarkeit des speziellen Halteproblems $H_\epsilon$ (siehe Vorlesung).